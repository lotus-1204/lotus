\documentclass[a4j]{jarticle}
%\usepackage{comment}
\usepackage{float}
\usepackage[dvipdfmx]{graphicx}
\usepackage{listings, jlisting}
\usepackage{url}
\lstset{
  basicstyle={\ttfamily},
  identifierstyle={\small},
  commentstyle={\smallitshape},
  keywordstyle={\small\bfseries},
  ndkeywordstyle={\small},
  stringstyle={\small\ttfamily},
  frame={tb},
  breaklines=true,
  columns=[l]{fullflexible},
  numbers=left,
  xrightmargin=0zw,
  xleftmargin=3zw,
  numberstyle={\scriptsize},
  stepnumber=1,
  numbersep=1zw,
  lineskip=-0.5ex
}


\title{情報学群実験第2 \\ ARMプロセッサのmov命令での \\ 即値のロード範囲の規則}

\author{学籍番号: 1240293 \\ 氏名:植田 蓮}

\date{\today}
\renewcommand{\lstlistingname}{実行例}
\begin{document}

\maketitle
\section{実験の目的}
本実験課題では,ARMプロセッサ用のアセンブリ言語(以下アセンブリ言語)を用いる.  
アセンブリ言語では,mov命令によってある範囲の即値を読み込むことができる.
しかし,その規則は単純ではない.例えば,256はロードできるが,257はロードすることができずアセンブルエラーとなる.
また,65280はロードすることができるが,65379や65281同じくアセンブルエラーとなる.本実験課題の目的は,ARMプロセッサのmov命令でロードできる値の範囲を表す規則を明らかにすることである.

\section{方法}
即値のロードできる規則を調べるために,今回の実験では,シェルスクリプトを用いて,
mov命令に0〜65535までの値をロードさせ,すべてアセンブルし,エラーになるのか否かを
調べ,その結果から,解析的に即値をロードできる範囲の規則を調べる.
用いたソースコードは付録1に示す.
また,レジスタは32ビットであることから,32ビットで表現されるいくつかの数値に対しても実験を行う.

\subsection{実行環境}
実験を行った環境を表1に示す
  \begin{table}[h]
    \centering
    \caption{実行環境}
    \begin{tabular}{|c|c|}
      \hline
      OS & Ubuntu 18.04.5 LTS 64bit \\ \hline
      shell & bash 4.4.20(1)-release (x86\_64-pc-linux-gnu) \\ \hline
    \end{tabular}
  \end{table}
  
  \subsection{実験に用いる32bitの値}
  今回実験には32bitで表現される値は$2147483648$,$2147483649$,$2281701377$,$2281709569$
  の4つを用いる.

  \subsection{シェルスクリプトの内部仕様}
  \verb+mov.s+には\verb+test_number+というターゲットが記述されている.
プログラムでは,このターゲットを\verb+sed+コマンドを用いて,0〜65535までの
値に書き換えたファイルを\verb+trash.s+として保存している.
そのファイルに対して,\verb+arm-none-eabi-as+を実行後\verb+echo $?+を行う.
アセンブルエラーなら,$1$が出力され,ロードに成功すると,$1$が出力される.
この出力をファイルに記述後に解析的に,即値をロードできる範囲の規則を導き出す.
また,解析のためにawkコマンドを用いて,\verb+echo $?+の結果が1の数値と$2$になる
数値を別のファイルに書き込んでいる.書き込んだファイルは付録2に示す.


\section{結果}
実験を行った結果,65536までの値については0〜256までの数値に関しては全てロードに成功した.
そして,260以上の値については,ロードに成功している数値は全て,偶数であることが分かった.
偶数がが全てロードに成功しているわけではないが,少なくとも257以上の奇数に関しては,
全ての値について,ロードに失敗している.
今回は65536個の数値に対して処理を行っているので,ここで
全ての結果を示すと膨大になるため,err.txt及びsuc.txtの一部を
付録2及び表2に示す.
\begin{table}[H]
  \centering
  \caption{実行結果}
  \begin{tabular}{|c|c|c|}
    \hline
    10進数 & 2進数 & アセンブル結果\\
    \hline
    256 & 100000000 & O \\
    \hline
    257 & 100000001 & X \\
    \hline
    516 & 1000000100 & O \\
    \hline
    517 & 1000000101 & X \\
    \hline
    519 & 1000000111 & X \\
    \hline
    1032 & 10000001000 & X \\
    \hline
    4128 & 1000000100000 & X \\
    \hline
    4200 & 1000100000000 & X \\
    \hline
    65280 & 1111111100000000 & O \\
    \hline
    65281 & 1111111100000001 & X \\
    \hline
    65535 & 1111111111111111 & X \\
    \hline
  \end{tabular}
\end{table}
また,32ビットで表現される値についての結果を表3に示す.
\begin{table}[H]
  \centering
  \caption{実行結果}
  \begin{tabular}{|c|c|c|}
    \hline
    10進数 & 2進数& アセンブル結果\\
    \hline
    2147483648 & 10000000000000000000000000000000 & O \\
    \hline
    2147483649 & 10000000000000000000000000000001 & O \\
    \hline
    2281701377 & 10001000000000000000000000000001 & O \\
    \hline
    2281709569 & 10001000000000000010000000000001 & X \\
    \hline

    
  \end{tabular}
\end{table}



\section{考察}
まずは実験結果において,0〜255までの値がロードできた理由について考察する.
$255=(2^{8}-1)$であることから,1バイトで表現できる数値に関してはmov命令ですべて
ロードできると考えられる.これはメモリが全て1バイト単位となっていることからも,
ごく自然なことである.\\
\indent 次に,256以上の値についても考える.0〜255とは違い,256は1バイトでは表現することができないにもかかわらず,
mov命令に即値として渡してもエラーとはならない.それに対し,257ではエラーとなる.
ここで2つの数値を2進数に変換する.
$256=100000000_{(2)}$であり,$257=100000001_{(2)}$である.
この2つの数値の大きな違いは2進数で表現したときに1が立っているビットが
8ビット以内に収まっているかどうかである.上述した0〜255は8ビット以内で
表現できるため,この規則に関しても必ず成り立っていると言える.
その他の数値に関しても,
$516=1000000100_{(2)}$は成功し,$517=1000000101_{(2)}$や$519=1000000111_{(2)}$は
即値のロードに失敗している.
更に大きな数値に関しても,$65280=1111111100000000_{(2)}$は成功し,$65281=1111111100000001_{(2)}$や$65535=1111111111111111_{(2)}$
はロードに失敗している.\\
\indent 一方で,$1032=10000001000_{(2)}$は1の立つビットが8ビット以内に
なっているにもかかわらず,エラーとなっている.同様にして,
$4128=1000000100000_{(2)}$もエラーとなっている.
これらの値の共通点は最上位ビットから8ビットに注目したときの最下位ビットから,0が奇数個連続になっていることである.\\
\indent ここからは,32ビットで表現される4つの値についての考察を行う.
$2147483648$については上述した規則で議論を行えば,ロードが成功することがわかる.
しかし$2147483649$,$2281701377$に関しては,こここまでで考えていた規則では,
エラーが起きるはずである.
ここで,これら2つの値と,エラーがおきた,$2281709569$を比較する.
この比較から,32ビットで考えた場合には,65535までで考えられる規則が,
回転シフトを行った際に成り立っていればロードできると考えられる.\\
\indent したがってmov命令がロードできる即値を表す規則は以下を満たすものだと考えられる.
\begin{itemize}
   \item 最上位ビットから8ビット以内にのみ,1が立っているビットが存在する
   \item 最上位ビットから8ビットに注目したとき,最下位ビットから0が偶数個連続である
   \item 数値を32ビットで表現したとき,任意の回転シフトを行うことで,上記2つの条件が満たされる
\end{itemize}
しかし,32ビットで表現される値については,用いたサンプルの数が非常に,
少ないので,さらなる検討が必要だと考えられる.


\newpage
\appendix
\renewcommand{\lstlistingname}{ソースコード}
\setcounter{lstlisting}{0}
\section{付録1}
\begin{lstlisting}[caption=test.sh]
  #!/bin/sh

count=0
while [ $count -lt 65536 ]
do
    sed -e "s/test_number/${count}/g" mov.s > trash.s
    echo -n $count >> log.txt
    arm-none-eabi-as trash.s
    echo $? >> log.txt
    count=`expr $count + 1`
done
awk '/1$/{print $1}' log.txt > err.txt
awk '/0$/{print $1}' log.txt > suc.txt
\end{lstlisting}

\begin{lstlisting}[caption=mov.s, label=mov]
    .section .text
    .global _start
_start:
    mov     r7, #1    @ exitのシステムコール番号
    mov     r0, #test_number  @ 終了コード
    swi     #0        @ システムコールの発行
\end{lstlisting}
\section{付録2}
\setcounter{lstlisting}{0}
\renewcommand{\lstlistingname}{結果}
\begin{lstlisting}[caption=suc.txt]
  //ロードに成功した値
  0:0
  1:0
  2:0
  .
  省略
  .
  255:0
  256:0 //256までの全ての数値はロードに成功
  260:0 //260以降は偶数のみ
  264:0
  .
  .
  516:0
  .
  .
  2064:0
  .
  .
  64768:0
  65024:0
  65280:0
  
  
\end{lstlisting}
\begin{lstlisting}[caption=err.txt]
// ロードに失敗した値
257:1
258:1
259:1
261:1
.n
省略
.
511:1
.
517:1
518:1
519:1
.
1032:1
.
.
4128:1
.
.
65534:1
65535:1
\end{lstlisting}
\begin{thebibliography}{9}
  \bibitem AArm, Documentation – Arm Developer, \url{https://developer.arm.com/documentation/dui0489/i/arm-and-thumb-instructions/mov}
\end{thebibliography}

\end{document}
